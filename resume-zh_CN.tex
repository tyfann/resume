% !TEX TS-program = xelatex
% !TEX encoding = UTF-8 Unicode
% !Mode:: "TeX:UTF-8"

\documentclass{resume}
\usepackage{zh_CN-Adobefonts_external} % Simplified Chinese Support using external fonts (./fonts/zh_CN-Adobe/)
%\usepackage{zh_CN-Adobefonts_internal} % Simplified Chinese Support using system fonts
\usepackage{linespacing_fix} % disable extra space before next section
\usepackage{cite}

\begin{document}
\pagenumbering{gobble} % suppress displaying page number

\name{唐宇帆}

% {E-mail}{mobilephone}{homepage}
% be careful of _ in emaill address
\contactInfo{(+31) 0645557124}{tangyf0522@gmail.com}{自然语言处理工程师}{GitHub @tyfann}
% {E-mail}{mobilephone}
% keep the last empty braces!
%\contactInfo{xxx@yuanbin.me}{(+86) 131-221-87xxx}{}
 
\section{个人总结}
本人在校成绩优秀、乐观向上,工作负责、自我驱动力强、热爱尝试新事物,认同人工智能在未来的不可替代性。在校期间长期从事自然语言处理以及高效的系统设计相关问题,对于自然语言处理文本研究相关的项目具有浓厚的兴趣和热情。\textbf{现就读于荷兰代尔夫特理工大学计算机科学专业硕士研究生}

% \section{\faGraduationCap\ 教育背景}
\section{教育背景}
\datedsubsection{\textbf{荷兰 代尔夫特理工大学},计算机科学,\textit{在读硕士研究生}}{2022.9 - 2024.6}
\ 预计2024年6月毕业
\datedsubsection{\textbf{南京航空航天大学},计算机科学与技术,\textit{工学学士}}{2018.9 - 2022.6}
\ \textbf{排名10/132(前8\%)},学业奖学金,优秀学生奖学金,三好学生
\datedsubsection{\textbf{英国 爱丁堡大学},信息学,\textit{公派交换生}}{2020.9 - 2020.12}
\ 

% \section{\faCogs\ IT 技能}
\section{技术能力}
% increase linespacing [parsep=0.5ex]
\begin{itemize}[parsep=0.2ex]
  \item \textbf{编程语言}: Java (Springboot), Python (Tensorflow, Pytorch), SQL, C, Shell
  \item \textbf{操作系统,数据库与工程构建}: Linux/macOS/MySQL/Redis/Git
\end{itemize}

% \end{itemize}

\section{实习经历}
\datedsubsection{\textbf{阿里云 | AliCloud}, JAVA研发工程师}{2021.7-2021.9}
\begin{itemize}
%   \item 飞猪北京前端团队全面负责各交通线的票务(机票/火车票/汽车票) web 应用与事业群基础架构研发
  \item 作为项目独立负责人之一开发监控告警管理平台Xdata,向业务的管理者以及应用的负责人展示业务与应用健康性能大盘。在这一过程中主要负责开发前端数据与数据库交互过程中会调用到的相应的增删改查功能接口,同时保证查询性能的稳定性
  \item 参与公司相关的销售生态管理业务,帮助开发面向销售团队的激励移动端,增加销售人员的政策参与体感,保证政策的落实度,从而保证公司销售的稳定增长。在这一过程中主要负责相应的数据库复杂查询接口的撰写与测试调优
\end{itemize}

\section{校内项目经历}
\datedsubsection{\textbf{Approximate NDCG Loss Optimization Exploration} 荷兰代尔夫特}{2023年3月 - 2023年4月}

研究Learning-to-rank相关的问题并在allrank模型上基于approxndcg,neuralndcg以及ranknet多个目标函数进行训练,汇报了它们在不同的指标(例如MRR, NDCG, DCG等)上的结果,并且针对approxndcg存在的相关问题进行了一系列的改进优化
\begin{onehalfspacing}
\begin{itemize}
    \item 在MQ2008和WEB10K上进行相关的实验研究,编写了多个自动化测试的脚本加快训练速度和效率
    \item 通过调整alpha以提升ApproxNDCG在WEB10K数据集上的性能
\end{itemize}
\end{onehalfspacing}


\datedsubsection{\textbf{Text-to-SQL研究与实现(本科毕业设计)} 江苏南京}{2021年12月 - 2022年6月}

基于开源的Text-to-SQL模型以及相关的电力领域Text-to-SQL数据集,首先完成了模型的复现,同时引入了数据增强的方法利用seq2seq的方法实现了对数据集的扩充,进而提升了baseline模型在当前数据集上的准确度
\begin{onehalfspacing}
\begin{itemize}
  \item 在中文的Text-to-SQL数据集上进行相关的研究,其中数据集涉及Chase、DuSQL、CSpider等等
  \item 在不额外增加人力资源的情况下,利用模版规则匹配的方法,对现有的数据集上进行相应的数据增强工作,同时利用机器翻译的方法和预训练的NLP模型例如bert、mBart、mT5等针对当前任务,在现有数据集上进行fine-tune,最终实现Text-to-SQL的模型
\end{itemize}
\end{onehalfspacing}

\section{竞赛获奖/项目作品}
% increase linespacing [parsep=0.5ex]
\begin{itemize}[parsep=0.2ex]
%   \item LeetCodeOJ Solutions, \textit{https://github.com/hijiangtao/LeetCodeOJ}
  \item 全国大学生数学建模竞赛江苏省赛区\textbf{二等奖},2020年12月
  \item 个人博客:\textit{https://tyfann.github.io/}
  % ,更多作品见 \textit{https://github.com/tyfann}
%   \item 电视节目"爸爸去哪儿"可视化分析展示, \textit{https://hijiangtao.github.io/variety-show-hot-spot-vis/}
\end{itemize}

% \section{\faHeartO\ 项目/作品摘要}
% \section{项目/作品摘要}
% \datedline{\textit{An Integrated Version of Security Monitor Vis System}, https://hijiangtao.github.io/ss-vis-component/ }{}
% \datedline{\textit{Dark-Tech}, https://github.com/hijiangtao/dark-tech/ }{}
% \datedline{\textit{融合社交网络数据挖掘的电视节目可视化分析系统}, https://hijiangtao.github.io/variety-show-hot-spot-vis/}{}
% \datedline{\textit{LeetCodeOJ Solutions}, https://github.com/hijiangtao/LeetCodeOJ}{}
% \datedline{\textit{Info-Vis}, https://github.com/ISCAS-VIS/infovis-ucas}{}


% \section{\faInfo\ 社会实践/其他}

%% Reference
%\newpage
%\bibliographystyle{IEEETran}
%\bibliography{mycite}
\end{document}
